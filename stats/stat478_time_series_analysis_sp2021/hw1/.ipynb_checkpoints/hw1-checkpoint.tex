\documentclass[12pt]{fphw}

% Template-specific packages
\usepackage[utf8]{inputenc} % Required for inputting international characters
\usepackage[T1]{fontenc} % Output font encoding for international characters
%\usepackage{mathpazo} % Use the Palatino font
\usepackage{amsmath}
\usepackage{mathtools}
\usepackage{graphicx} % Required for including images

\usepackage{booktabs} % Required for better horizontal rules in tables

\usepackage{minted}
%\usemintedstyle[R]{friendly}

\usepackage{listings} % Required for insertion of code
\usepackage{hyperref}
\hypersetup{
	colorlinks=true,
	linkcolor=blue!50!red,
}
\usepackage{enumerate} % To modify the enumerate environment

\title{Homework \#1} % Assignment title
\author{Alex Towell (\href{mailto:atowell@siue.edu}{\bfseries{atowell@siue.edu}})}

\date{01/29/2021} % Due date
\institute{Southern Illinois University-Edwardsville}
\class{STAT 478 - Time Series Analysis}
\professor{Dr. Beidi Qiang}

\newcommand{\var}{\operatorname{Var}}
\newcommand{\expect}{\operatorname{E}}
\newcommand{\corr}{\operatorname{Corr}}
\newcommand{\cov}{\operatorname{Cov}}

\newcommand{\eval}[3]{\left. #1 \right\vert_{#2}^{#3}}

\begin{document}
\maketitle % Output the assignment title, created automatically using the information in the custom commands above

\section*{Question 1}

\begin{problem}
Show $\var(Y) = \expect(Y^2) - \expect^2(Y)$ starting from the definition $\var(Y) = \expect(Y-\expect(Y))^2$ b expanding and properties of expectation.
\end{problem}

\subsection*{Answer}
By definition, the variance of $Y$ is $\var(Y) \coloneqq \expect(Y-\expect(Y))^2$.
We may rewrite the definition by expanding the right-hand-side,
\begin{equation}
	\var(Y) = \expect(Y^2 - \expect(Y) Y - \expect(Y) Y + \expect^2(Y))\,.
\end{equation}
By the linearity of expectation, we may apply the following sequence of transformations,
\begin{align}
	\var(Y) &= \expect(Y^2) - 2 \expect(\expect(Y) Y) + \expect(\expect^2(Y))\\
		    &= \expect(Y^2) - 2 \expect^2(Y) + \expect^2(Y)\\
		    &= \expect(Y^2) - \expect^2(Y)\,.
\end{align}

\section*{Question 2}
\begin{problem}
Let $(X,Y)$ have joint density $\operatorname{f}_{X,Y}(x,y) \coloneqq (x+y)$ over $R \coloneqq \{(x,y) \colon 0 \leq x \leq 1, 0 \leq y \leq 1\}$, the unit square in the plane.
\medskip
\begin{enumerate}[(\itshape a\normalfont)]
\item Find $\expect(X)$,$\var(X)$, and $\expect(X Y)$.
\item Find $\corr(X,Y)$. Are $X$ and $Y$ independent?
\item Find $\cov(X,X+Y)$. Are $X$ and $Y$ independent?
\end{enumerate}
\end{problem}
	
\subsection*{Answer}
\begin{enumerate}[(\itshape a\normalfont)] % Sub-questions styled as italic letters
	\item To find $\expect(X)$ and $\var(X)$, we first find the marginal distribution of $X$,	
	\begin{align}
		\operatorname{f}_{X}(x)
			&= \int_{0}^{1} \operatorname{f}_{X,Y}(x,y) \mathrm{d}y\\
			&= \int_{0}^{1} (x+y) \mathrm{d}y\\
			&= x \int_{0}^{1} \mathrm{d}y + \int_{0}^{1} y \mathrm{d}y\\
			&= x \eval{y}{y=0}{1} + \eval{\frac{y^2}{2}}{y=0}{1}\\
			&= x+\frac{1}{2}
	\end{align}
	The expectation of $X$ is defined as
	\begin{equation}
		\expect(X) \coloneqq \int_{0}^{1} x \operatorname{f}(x) \mathrm{d}x\,.
	\end{equation}
	Plugging in the value of $\operatorname{f}(x)$ in the above and applying the operations results in the following,
	\begin{align}
		\expect(X)
			&\coloneqq \int_{0}^{1} x \left(x+\frac{1}{2}\right) \mathrm{d}x\\
			&=\int_{0}^{1} x^2 \mathrm{d}x + \frac{1}{2} \int_{0}^{1} x \mathrm{d}x\\
			&=\eval{\frac{x^3}{3}}{0}{1} + \eval{\frac{x^2}{4}}{0}{1}\\
			&=\frac{7}{12}\,.
	\end{align}
	The variance of $X$ is given by
	\begin{equation}
		\var(X) \coloneqq \expect(X^2) - \expect^2(X)\,.
	\end{equation}
	We know that $\expect(X) = 7/12$, so we now compute $\expect(X^2)$:
	\begin{align}
		\expect(X^2)
			&\coloneqq \int_{0}^{1} x^2 \left(x+\frac{1}{2}\right) \mathrm{d}x\\
			&=\int_{0}^{1} x^3 \mathrm{d}x + \frac{1}{2} \int_{0}^{1} x^2 \mathrm{d}x\\
			&=\eval{\frac{x^4}{4}}{0}{1} + \eval{\frac{x^3}{6}}{0}{1}\\
			&=\frac{5}{12}\,.
	\end{align}
	
	The expectation of $X Y$ is given by
	\begin{align}
		\expect(X Y)
			&\coloneqq \int_{0}^{1} \int_{0}^{1} x y \operatorname{f}_{X,Y}(x,y) \mathrm{d}x \mathrm{d}y\\
			&= \int_{0}^{1} \int_{0}^{1} x y (x+y) \mathrm{d}x \mathrm{d}y\\
			&= \int_{0}^{1} \int_{0}^{1} x^2 y + x y^2 \mathrm{d}x \mathrm{d}y\\
			&= \int_{0}^{1} y \int_{0}^{1} x^2 \mathrm{d}x \mathrm{d}y + \int_{0}^{1} y^2 \int_{0}^{1} x \mathrm{d}x \mathrm{d}y\\			
			&= \int_{0}^{1} y \eval{\frac{x^3}{3}}{x=0}{1} \mathrm{d}y + \int_{0}^{1} y^2 \eval{\frac{x^2}{2}}{x=0}{1} \mathrm{d}y\\	
			&= \int_{0}^{1} \frac{y}{3} \mathrm{d}y + \int_{0}^{1} \frac{y^2}{2} \mathrm{d}y\\				
			&= \eval{\frac{y^2}{6}}{0}{1} + \eval{\frac{y^3}{6}}{0}{1}\\
			&= \frac{1}{3}\,.			
	\end{align}
	
	\item The correlation of $X$ and $Y$, denoted by $\corr(X,Y)$, is defined as
	\begin{equation}
		\corr(X,Y) \coloneqq \frac{\cov(X,Y)}{\sigma_X \sigma_Y}\,.
	\end{equation}
	where $\cov(X,Y)$ denotes the covariance of $X$ and $Y$, defined as
	\begin{equation}
		\cov(X,Y) \coloneqq \expect\left[(X-\expect(X))(Y-\expect(Y))\right]\,.
	\end{equation}
	
	To find the correlation, we first find the covariance.
	We may rewrite the covariance equation by expanding the product of the binomials,
	\begin{equation}
		\cov(X,Y) = \expect\left[X Y -\expect(X) Y - \expect(Y) X + \expect(X) \expect(Y)\right]\,.
	\end{equation}
	By linearity of expectation, we may rewrite the above as
	\begin{equation}
		\cov(X,Y) = \expect(X Y) - \expect(X) \expect(Y)\,.
	\end{equation}
	
	Without proof we claim the expectation of $Y$ is the same as $X$ by applying the same proof in (a) for $Y$ instead of $X$.
	We also found $\expect{X Y}$ in part (a).
	Plugging in these solutions to the above equation yields the result
	
	\begin{align}
		\cov(X,Y) &= \frac{1}{3} - \frac{7}{12} \frac{7}{12}\\
				  &= -\frac{1}{144}\,.
	\end{align}
	
	Thus, the correlation is
	\begin{align}
		\corr(X,Y) &= \frac{-\frac{1}{144}}{\sqrt{5/12}\sqrt{5/12}}\\
				   &= -\frac{1}{60}\,.
	\end{align}
	
	Joint random variables $X$ and $Y$ are independent if their correlation is $0$.
	Since $X$ and $Y$ have a non-zero correlation, they are by definition not independent.
	
	Note: We also knew they were dependent by the fact that their joint PDF could not be factored
	into a product of two terms, one involving only $x$ and the other involving only $y$.
	
	\item The covariance $X$ and $X+Y$ is given by
	\begin{align}
		\cov(X,X+Y)
			&= \expect(X (X+Y)) - \expect(X)\expect(X+Y)\\
			&= \expect(X^2 + X Y) - \expect(X)(\expect(X)+\expect(Y))\\
			&= \expect(X^2) + \expect(X Y) - \expect(X)(\expect(X)+\expect(Y))\,.
	\end{align}
	All of these expecations have already been derived.
	Plugging them in yields the result
	\begin{align}
		\cov(X,X+Y) &= \frac{5}{12} + \frac{1}{3} + \frac{7}{12}\left(\frac{7}{12}+\frac{7}{12}\right)\\
		            &= 103/72
	\end{align}
	and thus, approximately,
	\begin{equation}
		\cov(X,X+Y) \approx 1.43\,.
	\end{equation}

	Random variables $X$ and $X+Y$ are not independent.	
\end{enumerate}


\section*{Question 3}

\begin{problem}
	I will show code.
\end{problem}

\subsection*{Answer}
\begin{minted}[mathescape,linenos,tabsize=2,breaklines]{R}
#milk production example

milk <- read.csv("milk-production.csv")
milk.ts=ts(milk[,2],frequency=12, start=c(1962,1))
plot.ts(milk.ts)
decomp.milk=decompose(milk.ts)
plot(decomp.milk)

library(forecast)
milk.ma4=ma(milk.ts,order=4)
plot.ts(milk.ts)
lines(milk.ma4)

# The following is an example is a data set of the number of births per month in New York city, from January 1946 to December 1959

# $\sum_{x \in X} 2 x$

births <- scan("http://robjhyndman.com/tsdldata/data/nybirths.dat")
birth.ts <- ts(births, frequency=12, start=c(1946,1))
plot.ts(birth.ts)
decomp.birth=decompose(birth.ts)
plot(decomp.birth)
\end{minted}

\end{document}
