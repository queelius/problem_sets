\documentclass[10pt]{fphw}

% Template-specific packages
\usepackage[utf8]{inputenc} % Required for inputting international characters
\usepackage[T1]{fontenc} % Output font encoding for international characters
%\usepackage{mathpazo} % Use the Palatino font
\usepackage{amsmath}
\usepackage{mathtools}
\usepackage{graphicx} % Required for including images
\usepackage{amsthm}
\usepackage{commath}
\usepackage{multirow}
\usepackage{booktabs}
\usepackage{xcolor}
\renewcommand{\arraystretch}{1.2} 
%\usemintedstyle[R]{friendly}
\DeclareMathOperator*{\argmax}{arg\,max}

\usepackage{hyperref}
\hypersetup{
	colorlinks=true,
	linkcolor=blue!50!red,
}

\title{Homework \#4} % Assignment title
\author{Alex Towell (\href{mailto:atowell@siue.edu}{\bfseries{atowell@siue.edu}})}

\date{Feb 20, 2022} % Due date
\institute{Southern Illinois University-Edwardsville}
\class{STAT 579 - Discrete Multivariate Analysis}
\professor{Dr. Andrew Neath}

\newcommand{\odds}{\operatorname{odds}}
\renewcommand{\Pr}{\operatorname{P}}
\newcommand{\Eval}[3]{\left. #1 \right\vert_{#2}^{#3}}

\newcommand{\comment}[1]{}

\begin{document}
\maketitle % Output the assignment title, created automatically using the information in the custom commands above

\section*{Question 1}
Consider data from a retrospective study on the relationship between daily alcohol consumption and the
onset of esophagus cancer.

\begin{table}[h]
\centering
\begin{tabular}{@{}rrrr@{}}
\toprule
                                            &           & cancer&no cancer\\
\cmidrule{3-4}
\multirow{2}{*}{daily alcohol consumption}  & $> 80$g  	& 71    & 82\\
                                            & $< 80$g 	& 60    & 441\\
                                            & total     & 131   & 523\\
\bottomrule
\end{tabular}
\end{table}

\begin{enumerate}
\item[(a)] Compute an estimate of the odds ratio.
\item[(b)] Interpret the direction of the association, stated in the context of the problem.
\item[(c)] What assumption is necessary for the sample odds ratio to serve as an estimate of the relative risk?
\end{enumerate}  
\subsection*{Answer}

\begin{enumerate}
    \item[(a)] The odds ratio is defined as
    \begin{equation*}
    \theta \coloneqq \frac{\Omega_1}{\Omega_2}
    \end{equation*}
    where $\Omega_j$ is the odds of outcome of column $1$ given row $i$, which is defined as
    \begin{equation*}
    \Omega_j \coloneqq \frac{\pi_{1|i}}{1-\pi_{1|i}}\,.
    \end{equation*}
    
    Observe that the odds ratio has the interesting property of being invariant
    to the type of study, whether cross-sectional, prospective, or retrospective.
    An estimator of $\theta$ is given by
    \begin{align*}
    \hat\theta = \frac{n_{1 1} n_{2 2}}{n_{1 2} n_{2 1}} = \frac{71 \times 441}{ 82 \times 60} \approx 6.36\,.
    \end{align*}
    
    \item[(b)] Since $\hat\theta > 1$, we estimate that those consuming $>80$g of alcohol per day are more likely
    to have the onset of esophagus cancer than those who consume less than $80$g per day.
    
    \item[(c)] For the odds ratio $\theta$ to serve as a reasonable estimator of the relative risk,
    $\pi_{1|2}$ and $\pi_{1|1}$ must be close to zero since
    \begin{equation*}
        \theta = \operatorname{RR}\times\left(\frac{1-\pi_{1|2}}{1-\pi_{1|1}}\right)\,.
    \end{equation*}
\end{enumerate}  

\section{Question 2}
The following table summarizes the responses of $n=91$ couples to the questionnaire item
``Sex is fun for me and my partner.''
\begin{table}[h]
    \centering
    \begin{tabular}{@{}rrrrr@{}}
        \toprule
                              & \multicolumn{4}{c}{\bf{wife's rating}} \\
        \bf{husband's rating} & never/occassionally & farily often & very often & almost always\\
        never/occassionally   & 7 	              & 7    	     & 2          & 3\\
        farily often          & 2 	              & 8    	     & 3          & 7\\
        very often            & 1 	              & 5    	     & 4          & 9\\
        almost always         & 2 	              & 8    	     & 9          & 14\\
        \bottomrule
    \end{tabular}
\end{table}

\begin{enumerate}
    \item[(a)] What type of sampling was used in collecting the above data? What characteristic do the above variables
    have that allows us the potential to describe the association with a single parameter? In general, how many
    parameters would be needed to describe the association?
    \item[(b)] Compute $\hat\gamma$. Provide an interpretation of your result, stated in the context of the problem.
\end{enumerate}

\subsection*{Answer}


\begin{enumerate}
    \item[(a)] The data is a cross-sectional. The variables are \emph{ordinal} and the expected relationship
    is \emph{monotonic}. Ordinal variables have categories that may be placed in a natural \emph{order} and
    therefore support relational predicates like less-than in addition
    to equality.
    
    In general, $(4-1) \times (4-1) = 9$ parameters are necessary to describe all the possible
    associations.
    
    \item[(b)] The characteristic of interest, $\gamma$, is defined as
    \begin{equation*}
        \gamma \coloneqq \frac{\pi_C - \pi_D}{\pi_C + \pi_D}\,,
    \end{equation*}
    which is a measure of correlation for ordinal variates analagous to 
    Pearson's correlation coefficient.
    
    Modifying the R code by populating the \emph{counts} matrix with the data in the given 
    table, when I run the R program I get the following output:
    
\begin{verbatim}
                      wife's rating
husband's rating      never/occassionally fairly often very often almost always
  never/occassionally                   7            7          2             3
  fairly often                          2            8          3             7
  very often                            1            5          4             9
  almost always                         2            8          9            14
[1] 1508  709
[1] 0.368254 0.173138
[1] 0.36
\end{verbatim}
\end{enumerate}

The number of concordances is $C = 1508$ and the number of discordances is $D = 709$.
Estimators for $\pi_C$ and $\pi_D$ are respectively given by
\begin{equation*}
    \hat\pi_C = \frac{C}{\binom{n}{2}} = \frac{1508}{\binom{91}{2}} \approx .368
\end{equation*}
and
\begin{equation*}
    \hat\pi_D = \frac{D}{\binom{n}{2}} = \frac{709}{\binom{91}{2}} \approx .173\,.
\end{equation*}

An estimator for $\gamma$ is given by
\begin{equation*}
    \hat\gamma = \frac{\hat\pi_C - \hat\pi_D}{\hat\pi_C + \hat\pi_D} = \frac{0.368-0.173}{0.368+0.173} \approx 0.36\,.
\end{equation*}

We say that this value of gamma represents a \emph{medium effect}.
Thus, we estimate that there is a medium size, positive association
between the husband's and wife's rating of sex.
\end{document}
