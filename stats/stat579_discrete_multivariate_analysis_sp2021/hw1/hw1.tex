\documentclass[12pt]{fphw}[final]

% Template-specific packages
\usepackage[utf8]{inputenc} % Required for inputting international characters
\usepackage[T1]{fontenc} % Output font encoding for international characters
%\usepackage{mathpazo} % Use the Palatino font
\usepackage{amsmath}

\usepackage{graphicx} % Required for including images

\usepackage{booktabs} % Required for better horizontal rules in tables

\usepackage{listings} % Required for insertion of code

\usepackage{enumerate} % To modify the enumerate environment
\usepackage{hyperref}
\hypersetup{
    colorlinks=true,
    linkcolor=blue!50!red,
}

\title{Homework \#1} % Assignment title
\author{Alex Towell (\href{mailto:atowell@siue.edu}{atowell@siue.edu})} % Student name
%\email{atowell@siue.edu}
\date{01/28/2021} % Due date
\institute{Southern Illinois University-Edwardsville}
\class{STAT 579 - Discrete Multivariate Analysis}
\professor{Dr. Andrew Neath}

\begin{document}
\maketitle % Output the assignment title, created automatically using the information in the custom commands above
	
\section*{Question 1}

\begin{problem}
Consider repeated, independent rolls of a fair die.
\medskip
\begin{enumerate}[(\itshape a\normalfont)] % Sub-questions styled as italic letters
	\item Let $Y_1$ be the number of ones in $50$ rolls.
	Completely specify the probability function for $Y_1$.
	\item Let $(Y_1,Y_2,\ldots,Y_6)$ be the number of ones, twos, etc. in $50$ rolls.
	Completely specify the probability function for $(Y_1,Y_2,\ldots,Y_6)$.
\end{enumerate}
\end{problem}

\subsection*{Answer}
\begin{enumerate}[(\itshape a\normalfont)] % Sub-questions styled as italic letters
	\item Each roll is an independent trial. We model each roll as a discrete uniform distribution, $X_j \sim \mathrm{DU}(1,6)$ with probability distribution function (PDF)
	\begin{equation}
		\operatorname{f}_{X_i}(k) = \frac{1}{6}, k \in \{1,\ldots,6\}
	\end{equation}
    for $i=1,\ldots,50$.
	
	The probability that $X_i = 1$ is $\pi = \operatorname{f}_{X_i}(1) = \frac{1}{6}$ and the probability that $X_i \neq 1$ is $1-\pi = \frac{5}{6}$.
	$Y_j$ is the number of outcomes where $X_i = j$, i.e., $Y_j = \sum_{i=1}^{50} [X_i = j]$. Since each $X_i$ is i.i.d., $Y_j$ is binomially distributed as
	\begin{equation}
		Y_j \sim \mathrm{BIN}\left(n = 50, \pi = \frac{1}{6}\right)\,,
	\end{equation}
	with the PDF
	\begin{equation}
		\operatorname{f}_{Y_j}(k \,\vert\, n = 50, \pi = 1/6) = {50 \choose k} \left(\frac{1}{6}\right)^k \left(\frac{5}{6}\right)^{50-k}\,,
	\end{equation}
    
    Thus, $Y_1 \sim \operatorname{f}_{Y_1}(k \, \vert \, n = 50, \pi = 1/6 )$.
    
%	which may be rewritten as
%	\begin{equation}
%		\operatorname{f}_{Y_1}(k \,\vert\, n = 50, \pi = 1/6) = \left(\frac{5}{6}\right)^{50} {50 \choose k} 5^{-k}\,.
%	\end{equation}
	
	\item A single observation of $X_i$ may be considered as a random Boolean vector of dimension $6$ whose
    $j$-th component realizes $1$ if $X_i = j$ and otherwise $0$, e.g., $X_i = 3$ maps to $(0,0,1,0,0,0)$.
    Then, $\sum_{i=1}^{50} X_i$ maps to a random vector $(Y_1,\ldots,Y_6)$ where $Y_j = k_j$ if $k_j$ of $\{X_i\}$ realizes $j$.
    
    This joint distribution $(Y_1,\ldots,Y_6)$ models the multinomial distribution,
	\begin{equation}
		(Y_1,\ldots,Y_6) \sim \mathrm{MULT}(n=50,\{\pi_j\})
	\end{equation}
    where $\pi_j = 1/6$ for $j=1,\ldots,6$, which has the PDF
	\begin{equation}
		\operatorname{f}(k_1,\ldots,k_6) = \frac{50!}{\prod_{j=1}^6 k_j!} \prod_{j=1}^{6} \left(\frac{1}{6}\right)^{k_j}
	\end{equation}
	with the constraint that $\sum_{j=1}^{6} k_j = n = 50$. We may rewrite the above equation as
	\begin{equation}
		\operatorname{f}(k_1,\ldots,k_6) = \frac{50!}{\prod_{j=1}^6 k_j!} \left(\frac{1}{6}\right)^{\sum_{j=1}^{6} k_j}\,.
	\end{equation}	
	Since $k_1+\ldots+k_6 = 50$, we may rewrite the above equation as
	\begin{equation}
		\operatorname{f}(k_1,\ldots,k_6) = \frac{50!}{6^{50} \prod_{j=1}^6 k_j!}\,.
	\end{equation}
	Also, observe that $k_6 = 50 - k_1 - \ldots - k_5$, and thus we may reparamerterize as
	\begin{equation}
		\operatorname{f}(k_1,\ldots,k_5) = \frac{50!}{6^{50} (50 - \sum_{j=1}^5 k_j)\prod_{j=1}^5 k_j!}\,.
	\end{equation}
\end{enumerate}

\section*{Question 2}
\begin{problem}
Let $Y_1 \sim \mathrm{POI}(\lambda_1=1)$, $Y_2 \sim \mathrm{POI}(\lambda_2=2)$, and $Y_3 \sim \mathrm{POI}(\lambda_3=3)$ be independent random variables.
	
\medskip
\begin{enumerate}[(\itshape a\normalfont)]
\item Completely specify the probability function for $(Y_1, Y_2, Y_3)$.
\item Completely specify the probability function for $Y_{+} = \sum_{i=1}^{3} Y_i$.
\item Completely specify the conditional probability function for $(Y_1,Y_2,Y_3)$ given $Y_{+} = n$.
\end{enumerate}
\end{problem}
	
\subsection*{Answer}
\begin{enumerate}[(\itshape a\normalfont)] % Sub-questions styled as italic letters
	\item By independence,
    \begin{equation}
        \operatorname{f_{Y_1,Y_2,Y_3}}(k_1,k_2,k_3 \,\vert\, \lambda_1\!=\!1,\lambda_2\!=\!2,\lambda_3\!=\!3)
            = \prod_{i=1}^{3} \operatorname{f_{Y_i}}(k_i \,\vert\, \lambda_i)\,.
    \end{equation}
	Dropping the subscripts on $\operatorname{f}$ for notational simplicity and substituting the \emph{Poisson} probability distribution functions into the above equation, we rewrite the joint distribution function as
	\begin{equation}
		\operatorname{f}(k_1,k_2,k_3 \,\vert\, \lambda_1\!=\!1,\lambda_2\!=\!2,\lambda_3\!=\!3) = \frac{\lambda_1^{k_1} \mathrm{e}^{-k_1}}{k_1!}\frac{\lambda_2^{k_2} \mathrm{e}^{-k_2}}{k_2!}\frac{\lambda_3^{k_3} \mathrm{e}^{-k_3}}{k_3!}
	\end{equation}
	which may be rewritten as
	\begin{equation}
		\operatorname{f}(k_1,k_2,k_3 \,\vert\, \lambda_1\!=\!1,\lambda_2\!=\!2,\lambda_3\!=\!3) =
        \frac{\lambda_1^{k_1}\lambda_2^{k_2}\lambda_3^{k_3} \mathrm{e}^{-(k_1+k_2+k_3)}}{k_1! k_2! k_3!}\,.
	\end{equation}
	
	We may rewrite the above by substituting the given lambda values into right-hand-side,
	\begin{equation}
		\operatorname{f}(k_1,k_2,k_3) = \frac{2^{k_2}3^{k_3} \mathrm{e}^{-(k_1+k_2+k_3)}}{k_1! k_2! k_3!}
	\end{equation}
	
	\item The sum of independent poisson random variables is poisson with a failure rate given by the sum of the poisson failure rates, thus
	\begin{equation}
		Y_{+} \sim \mathrm{POI}(\lambda_{+} = \lambda_1 + \lambda_2 + \lambda_3 = 6)\,,
	\end{equation}
	which has the PDF
	\begin{equation}
		\operatorname{f}_{Y_{+}}(n \,\vert\, \lambda = 6) = \frac{6^{n} \mathrm{e}^{-n}}{n!}
	\end{equation}
	where $k \in \{0,1,2,\ldots\}$.
	
	\item By the laws of probability, the conditional distribution of $(Y_1,Y_2,Y_3)$ given $Y_{+} = n$ is
	\begin{align}
		\operatorname{P}(Y_1 = k_1,Y_2 = k_2,Y_3 = k_3 &\,\vert\, Y_{+} = n) =\\
			&\frac{\operatorname{P}(Y_1 = k_1,Y_2 = k_2,Y_3 = k_3, Y_{+} = n)}{\operatorname{P}(Y_{+} = n)}\,.
	\end{align}
	It is given that $k_1+k_2+k+k_3 = n$, and therefore the only value that $Y_{+}$ can realize is $n$ with
	probability $1$, therefore
	\begin{align}
		\operatorname{P}(Y_1 = k_1,Y_2 = k_2,Y_3 = k_3 &\,\vert\, Y_{+} = n) =\\
		&\frac{\operatorname{P}(Y_1 = k_1,Y_2 = k_2,Y_3 = k_3)}{\operatorname{P}(Y_{+} = n)}\,.
	\end{align}
	Plugging in the joint distribution function for $(Y_1,Y_2,Y_3)$ and the distribution function for $Y_{+}$ results in
	\begin{equation}
		\operatorname{P}(Y_1 = k_1,Y_2 = k_2,Y_3 = k_3 \,\vert\, Y_{+} = n) =
            \frac{\operatorname{f}(k_1,k_2,k_3 \,\vert\, \lambda_1,\lambda_2,\lambda_3)}{\operatorname{f_{Y_{+}}}\!(n\,\vert\,\lambda_{+})}\,.
	\end{equation}	
	We may rewrite the above by plugging in the derivations of these distribution functions,
	\begin{equation}
        \operatorname{P}(Y_1 = k_1,Y_2 = k_2,Y_3 = k_3 \,\vert\, Y_{+} = n) = \frac{ n! \lambda_1^{k_1}\lambda_2^{k_2}\lambda_3^{k_3} \mathrm{e}^{-(k_1+k_2+k_3)}}{6^n \mathrm{e}^{-n} k_1! k_2! k_3!}
    \end{equation}	
    Since it is given that $k+1+k_2+k_3 = n$, we may perform these substitutions as desired, resulting in sequence of transformations given by
	\begin{align}
        \operatorname{P}(Y_1 = k_1,Y_2 = k_2,Y_3 = k_3 \,\vert\, Y_{+} = n)
            &= \frac{ n! \lambda_1^{k_1}\lambda_2^{k_2}\lambda_3^{k_3} \mathrm{e}^{-n}}{6^n \mathrm{e}^{-n} k_1! k_2! k_3!}\\
            &= \frac{ n! \lambda_1^{k_1}\lambda_2^{k_2}\lambda_3^{k_3}}{6^n k_1! k_2! k_3!}\\
            &= \frac{ n! \lambda_1^{k_1}\lambda_2^{k_2}\lambda_3^{k_3}}{6^{k_1+k_2+k_3} k_1! k_2! k_3!}\\
            &= \frac{ n! \lambda_1^{k_1}\lambda_2^{k_2}\lambda_3^{k_3}}{6^{k_1}6^{k_2}6^{k_3} k_1! k_2! k_3!}\\
            &= \frac{ n! }{k_1! k_2! k_3!}
                \!\left(\frac{\lambda_1}{6}\right)^{\!k_1}
                \!\!\left(\frac{\lambda_2}{6}\right)^{\!k_2}
                \!\!\left(\frac{\lambda_3}{6}\right)^{\!k_3}\,,
    \end{align}	
	which is the distribution function of the multinomial.
    
    If we let $W$ denote the conditional distribution of $(Y_1,Y_2,Y_3)$ given $Y_{+} = n = 50$, then
    \begin{equation}
         W \sim \mathrm{MULT}(n=50, \{\pi_j\})
    \end{equation}
    where $\pi_j = \frac{\lambda_j}{\lambda_{+}} = \frac{j}{6}$. This may finally be rewritten to
    \begin{equation}
        W \sim \mathrm{MULT}\!\left(n=50, \pi_1=\frac{1}{6}, \pi_2=\frac{2}{6}, \pi_3=\frac{3}{6}\right)\,.
    \end{equation}
\end{enumerate}
	
\end{document}
